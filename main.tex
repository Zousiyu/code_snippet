\documentclass[oneside]{book}

%% 必须的包
\usepackage{amsmath}
\usepackage{amsfonts}
\usepackage{amssymb}
\usepackage{xcolor}

%% 中文文字处理
\colorlet{title}{blue!40!black}
\usepackage[heading = true]{ctex}
    \ctexset{today = old}
    \pagestyle{plain}  % 自定义板式
    %% 标题设置
    \ctexset{
        chapter = {
            pagestyle = chapterpage,
        },
        section = {
            name = {问题},
            format = \bf\raggedright\color{title}\zihao{4},
            numberformat = \rmfamily,
            number = \arabic{section},
        },
        subsection = {
            format = \rmfamily\heiti\raggedright\color{title}\zihao{5},
        },
    }%

%% 字体设置
\usepackage{fontspec}
    \setmainfont{Adobe Garamond Pro}  % TeX Gyre Pagella
    \setsansfont{TeX Gyre Heros}
    \setmonofont{Fira Code}  % Source Code Pro, Consolas, DejaVu Sans Mono
    \setCJKmainfont[BoldFont={Source Han Sans SC}, ItalicFont={KaiTi}]{Source Han Serif SC}
    \setCJKmonofont{Inziu Iosevka Slab SC}  % FangSong
    \setCJKsansfont{Source Han Sans SC}

%% 数学字体
\usepackage{unicode-math}
    \setmathfont[
        math-style = ISO,
        bold-style = ISO
        ]{TeX Gyre Pagella Math}

\usepackage{geometry}
    \geometry{paper=a4paper,
        hmargin = 3cm,
        vmargin = 2cm,
        marginparwidth = 2.5cm,
    }%

%% 代码展示
\definecolor{bg}{rgb}{0.95, 0.95, 0.95}

\usepackage[newfloat=true]{minted}
    \usemintedstyle{colorful}
    \newminted{Matlab}{bgcolor=bg, breaklines=true}  % Matlabcode环境
    \newminted{python}{bgcolor=bg}
    \newminted{TeX}{bgcolor=bg, breaklines=true}
    %% 行内代码
    \newmintinline{Matlab}{}
    %% 格式化文件
    \newmintedfile{Matlab}{bgcolor=bg, breaklines=true}

%% 算法展示
\usepackage[lined, algoruled, algochapter]{algorithm2e}
    \SetKwProg{Function}{Function}{:}{end}


%% 页眉页脚
\usepackage{fancyhdr}
\pagestyle{fancy}
    \fancyhead{}
        \lhead{\kaishu\nouppercase{\leftmark}}
        \rhead{\thepage}
    \fancyfoot{}
% 章标题所在页专用版式
\fancypagestyle{chapterpage}{%
    \chead{\slshape\leftmark}
    \lhead{\sffamily 代码笔记本}
    \rhead{\thepage}
}

%% 脚注,带圈数字,刘海洋,可以超过10
\usepackage{xunicode-addon}
\newfontfamily\fnmarkfont{ipag.ttf} % 带圈 0 到 20 被认做西文符号
\newCJKfontfamily\fnCJKmarkfont{ipag.ttf} % 带圈数字超过 20 是 CJK 符号
\renewcommand\thefootnote{
    {\fnmarkfont\fnCJKmarkfont\textcircled{\arabic{footnote}}}
}

\usepackage{enumitem}%列表
    \setlist[itemize]{%
        nosep,
        leftmargin=\parindent,
}

%% 自定义命令
\newcommand*{\algoref}[1]{算法\ref{#1}}

%% 超链接
\usepackage{hyperref}
    \hypersetup{%
        bookmarksopen=true,  % 展开书签
        bookmarksnumbered=true,  % 显示书签编号
        bookmarksopenlevel=1,
        unicode=true,  % 使书签支持unicode字符
        %链接、颜色
        breaklinks=true,  % 链接自动换行
        colorlinks=true,  % 加颜色区分链接
        citecolor=black,  % 文献序号颜色
    }
    %定制pdf属性
    \hypersetup{%
        pdftitle={代码笔记本},
        pdfauthor={邹思宇},
        pdfkeywords={LaTeX, PGFplot, Tikz, MATLAB, C, Python},
        pdfstartview=Fit,%整个页面适合窗口
        pdfcreator={XeLaTeX \& TeXStudio}
    }%

%% url样式
\newfontfamily\urlfont{PT Sans Narrow}
\def\UrlFont{\urlfont}
\urlstyle{urlfont}

\begin{document}
    \frontmatter
    
    \mainmatter
    \chapter*{导言}

这份文档主要用来存放一些实际工作中碰到的实用的代码片段,可能包含MATLAB、Python、C/C++和一些\LaTeX{}的小知识。个人笔记,个人娱乐。

如果有人想编译这份手册或想学习一下实现,请务必读以下说明。

\textbf{字体设置},为了避免侵权,尽可能使用开源字体\footnote{西文主字体Adobe Garamond Pro、楷体、仿宋暂时没有找到理想的替代方案}。

\begin{itemize}
\item Source Han Sans: \url{https://github.com/adobe-fonts/source-han-sans/tree/release}
\item Source Han Serif: \url{https://github.com/adobe-fonts/source-han-serif/tree/release}
\item Source Code Pro: \url{https://github.com/adobe-fonts/source-code-pro}
\item PT Sans Narrow: \url{https://fonts.google.com/specimen/PT+Sans+Narrow}
\item TeX Gyre: 有问题前往\url{https://www.ctan.org}获取,一般来说\TeX{}发行版自带
\item 等宽字体:大多数等宽字体都是程序员使用的,开源居多,颇易获取。我常用DejaVu Sans Mono,Fira Code和Source Code Pro三种。
\end{itemize}

\begin{minted}{TeX}
%% 字体设置
\usepackage{fontspec}
    \setmainfont{Adobe Garamond Pro}  % TeX Gyre Pagella
    \setsansfont{TeX Gyre Heros}
    \setmonofont{Source Code Pro}  % Consolas, DejaVu Sans Mono
    \setCJKmainfont[BoldFont={Source Han Sans SC}, ItalicFont={KaiTi}]{Source Han Serif SC}
    \setCJKmonofont{FangSong}
    \setCJKsansfont{Source Han Sans SC}

%% 数学字体
\usepackage{unicode-math}
    \setmathfont[math-style = ISO, bold-style = ISO]{TeX Gyre Pagella Math}

%% url样式
\newfontfamily\urlfont{PT Sans Narrow}
\end{minted}

\textbf{编译环境设置},代码高亮环境由minted宏包提供(需要Python环境)。

\textbf{代码测试环境},各种代码的运行环境为MATLAB 2017b、Anaconda、Visual Studio 2017 community、MiK\TeX{}(各宏包均为最新)。

如果你觉得本文档里的代码有用,请不要直接复制文档里面的代码(直接复制会复制到换行产生的符号及空格,可能导致代码出现难以预计的错误),请到\href{https://github.com/Zousiyu/code_snippet}{github项目}的code文件夹找对应的文件。
    \chapter{MATLAB}

\section{如何遍历当前文件夹及其子文件夹中的全部文件}

假设现在我们有这样一个文件夹A,它含有一些文件和子文件夹B、C、D......,这些子文件夹有包含若干层子文件夹。我们需要将这个父文件夹(A)及其子文件夹(B、C、D......)和孙文件夹中的所有文件名和其路径取出来。

如果你用的是MATLAB 2016b及其后面的版本,那真的太棒了!\Matlabinline|dir()|函数已经支持遍历搜索了。尝试敲入:

\begin{Matlabcode}
dir_data = dir('**/*');
dir_data([dir_data.isdir]) = [];  % 去除所有文件夹
\end{Matlabcode}

这将会返回一个包含文件信息的struct,现在你可以任意操作这些struct了,随意拼接路径。解放大脑,哦也!

方便归方便,但是,一来肯定有大多数人使用的是MATLAB 2016b之前的版本,二来,解放大脑意味着我失去了一次独立思考的机会。

\subsection*{思考}

对于实现方法\footnote{思路来源:\href{https://stackoverflow.com/questions/2652630/how-to-get-all-files-under-a-specific-directory-in-matlab}{How to get all files under a specific directory in MATLAB?}},多层次的遍历,我们第一时间想到的肯定是递归。然后就是数据的存储了,\Matlabinline|dir()|函数返回的是一个struct,这个数据结构储存有文件的name和folder,我们要充分利用这个数据结构。所以现在思路是,写一个递归函数,这个函数返回包含所有文件的struct。

这个函数应对先处理父文件夹,获取文件和子文件夹,在获取子文件夹的过程中,我们需要去除`.'和`..'这两个特殊的文件夹。我们对获取的子文件夹再次调用该函数。如此,利用递归获取子子孙孙无穷尽文件夹的信息\footnote{其实这并不可能,因为递归是有栈高度限制的,调用函数压入栈,返回函数弹出栈,如果文件夹层次太深,一直压栈就会到达栈溢出警告的极限,例如Python的栈往往是100层,我想MATLAB的栈也大致如此,不会太高},最后函数返回存储有所有文件信息的struct。现在,你可以对这个结构体做你想做的事情。思路如\algoref{get_all_file_name}所示。

\begin{algorithm}[H]
\KwIn{path}
\KwOut{struct of file information}
\Function{get\_all\_file\_name(path)}{
    get file and sub\_dir information of current dir\;
    storing file information\;
    remove specific folder\;
    \BlankLine
    \BlankLine
    \For{first sub\_dir \KwTo last sub\_dir)}{
        get next sub\_dir\;
        \textbf{recursion} $ \Longrightarrow $ get\_all\_file\_name(path)\;}
%    \KwRet{struct of file information}\;
}
\caption{遍历获取当前文件夹及其所有子文件夹中的文件名}
\label{get_all_file_name}
\end{algorithm}

\subsection*{解}

MATLAB 2016b以上的版本我们可以用函数返回struct,这个数据结构包含[folder, name, date, bytes, isdir, datenum]六个字段的信息,我们可以按自己意愿使用folder和name拼接出文件的完整路径。

\Matlabfile[firstline=1]{code/get_all_file_name_R2016b_newer.m}

2016a及之前的版本dir struct信息并不包含folder,如果返回struct,将只有文件的[name, date, bytes, isdir, datenum]五个字段的信息,所以我们并不能根据函数返回的struct拼接出文件完整路径,\textbf{我们需要自己将路径拼接成一个cell,然后使用函数返回cell}。

\Matlabfile{code/get_all_file_name_R2016a_older.m}

\subsection*{总结}
\Matlabinline|dir()|函数遍历整个F盘共2万余文件文件大约需要1.555823s。我们实现的递归函数遍历F盘文件大约需要3.703009s。慢是慢了点,但我们成功运用了递归解决问题,不是吗?

\section{title中英文标题}
    \chapter{Python}

\section{如何展开一个嵌套的序列?}

我们现在有这样一个序列\codeinline{python}{items = [1, 2, [3, 4, [5, 6, [9, 8], 7], 8]]},我们想逐级展开这个序列,然后将所有元素装入一个序列。

如果这个序列层级较少,我们可以用多层\codeinline{python}{for}循环来遍历这个序列。一旦这个序列超过3层,过多的循环会让你很头疼。同样,这种多层级的问题我们可以用\textbf{递归}来解决。构建一个函数,这个函数能处理第一层的元素,由于第二层是\codeinline{python}{list},它是一个可迭代对象,我们只需要判断第二层是不是可迭代对象,同时忽略\codeinline{python}{str, bytes}对象\footnote{\codeinline{python}{str, bytes}也是可迭代对象,我们要避免其展开成单个字符。}。只要内层是可迭代的,我们就开始递归,对其应用该函数。

\pythonfile{code/python/unfold.py}

由于存在递归,所以函数会被调用很多次,每次调用所得的数据都需要保留,如何在多次的调用之间共享保留数据呢?我采用一个默认参数来实现,首次调用时不给默认参数新值,这会产生一个空的\codeinline{python}{list},当对内层对象调用时,将上一次产生的数据赋值给这个参数。输出结果:

\begin{pythoncode}
>>> items1 = ['Paula', ['Thomas', 'Lewis', ['siyu', 'ziyan', ['jianyuan']]]]
>>> items2 = [1, 2, [3, 4, [5, 6, [9, 8], 7], 8]]
>>> items3 = [[1, 2], 3, (4, [5, 6])]
>>> print(unfold(items1))
>>> print(unfold(items2))
>>> print(unfold(items3))
['Paula', 'Thomas', 'Lewis', 'siyu', 'ziyan', 'jianyuan']
[1, 2, 3, 4, 5, 6, 9, 8, 7, 8]
[1, 2, 3, 4, 5, 6]
\end{pythoncode}

但这样做有两个显而易见的坏处,一是当我们的嵌套序列有无限多层,递归会栈溢出;二是序列整个被读取到内存中了,当序列元素非常多,比如1亿,内存会被撑死。坏处一我们不去管他,大多数情况下是适用的,坏处二可以很容易的利用generator来解决\footnote{思路来源~\url{http://python3-cookbook.readthedocs.io/zh_CN/latest/c04/p14_flattening_nested_sequence.html}}。

\pythonfile{code/python/unfold_generator.py}

使用generator一来能防止内存爆炸,二来不需要在函数的多次调用见传递数据,代码更清晰明朗。需要注意,generator是惰性序列,边调用边计算,我们需要使用\codeinline{python}{for}迭代出每一个元素或者直接用\codeinline{python}{list()}获取全部元素。

\begin{pythoncode}
items1 = ['Paula', ['Thomas', 'Lewis', ['siyu', 'ziyan', ['jianyuan']]]]
items2 = [1, 2, [3, 4, [5, 6, [9, 8], 7], 8]]
items3 = [[1, 2], 3, (4, [5, 6])]
print(list(unfold(items1)))
print(list(unfold(items2)))
print(list(unfold(items3)))
\end{pythoncode}
    \chapter{算法}

\section{简单算法}
    \backmatter
\end{document}