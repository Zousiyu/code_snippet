\chapter{C和C++}

\section{C语言的动态数组}

大多数时候为了方便(其实是我菜),会使用库较多(方便)的C++,但是C语言在实际生产中使用率仍然很高,比如长期使用的ANSYS Fluent的UDF就不得不用C语言。下面是一个简易的动态数组的实现,来源\footnote{\url{https://stackoverflow.com/questions/3536153/c-dynamically-growing-array}}。

\begin{minted}{c}
#include <iostream>

typedef struct
{
    int *array;
    size_t used;
    size_t size;
} Array;

void initArray(Array *a, size_t initialSize)
{
    a->array = (int *)malloc(initialSize * sizeof(int));
    a->used = 0;
    a->size = initialSize;
}

void insertArray(Array *a, int element)
{
    // a->used is the number of used entries, because a->array[a->used++] updates a->used only *after* the array has been accessed.
    // Therefore a->used can go up to a->size
    if (a->used == a->size)
    {
        a->size *= 2;
        a->array = (int *)realloc(a->array, a->size * sizeof(int));
    }
    a->array[a->used++] = element;
}

void freeArray(Array *a)
{
    free(a->array);
    a->array = NULL;
    a->used = a->size = 0;
}

int main()
{
    Array a;
    int i;

    initArray(&a, 5); // initially 5 elements
    for (i = 0; i < 100; i++)
        insertArray(&a, i);     // automatically resizes as necessary
    printf("%d\n", a.array[9]); // print 10th element
    printf("%d\n", a.used);     // print number of elements
    freeArray(&a);
    return 0;
}

\end{minted}