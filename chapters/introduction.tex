\chapter*{导言}

这份文档主要用来存放一些实际工作中碰到的实用的代码片段,可能包含MATLAB、Python、C和一些\LaTeX{}的小知识。由于我是一个化学工程的学生,同时也初涉编程未深,计算机专业知识难免会出点错误,欢迎指正。

如果有人想编译这份手册或想学习一下实现,请务必读以下说明。

\textbf{字体设置},为了避免侵权,尽可能使用开源字体\footnote{西文主字体Adobe Garamond Pro、楷体、仿宋暂时没有找到理想的替代方案}。

\begin{itemize}
\item Source Han Sans: \url{https://github.com/adobe-fonts/source-han-sans/tree/release}
\item Source Han Serif: \url{https://github.com/adobe-fonts/source-han-serif/tree/release}
\item Source Code Pro: \url{https://github.com/adobe-fonts/source-code-pro}
\item PT Sans Narrow: \url{https://fonts.google.com/specimen/PT+Sans+Narrow}
\item TeX Gyre: 有问题前往\url{https://www.ctan.org}获取,一般来说\TeX{}发行版自带
\item 等宽字体:大多数等宽字体都是程序员使用的,开源居多,颇易获取。我常用DejaVu Sans Mono,Fira Code和Source Code Pro三种。
\end{itemize}

\begin{TeXcode}
%% 字体设置
\usepackage{fontspec}
    \setmainfont{Adobe Garamond Pro}  % TeX Gyre Pagella
    \setsansfont{TeX Gyre Heros}
    \setmonofont{Source Code Pro}  % Consolas, DejaVu Sans Mono
    \setCJKmainfont[BoldFont={Source Han Sans SC}, ItalicFont={KaiTi}]{Source Han Serif SC}
    \setCJKmonofont{FangSong}
    \setCJKsansfont{Source Han Sans SC}

%% 数学字体
\usepackage{unicode-math}
    \setmathfont[math-style = ISO, bold-style = ISO]{TeX Gyre Pagella Math}

%% url样式
\newfontfamily\urlfont{PT Sans Narrow}
\end{TeXcode}

\textbf{编译环境设置},代码高亮环境由minted宏包提供(需要Python环境)。

\textbf{代码测试环境},各种代码的运行环境为MATLAB 2017a、Anaconda 4.4.0(Python 3.6x)、Visual Studio 2017 community、MiK\TeX{} 2.9(各宏包均为最新)。