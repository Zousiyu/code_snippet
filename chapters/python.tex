\chapter{Python}

\section{如何展开一个嵌套的序列?}

我们现在有这样一个序列\codeinline{python}{items = [1, 2, [3, 4, [5, 6, [9, 8], 7], 8]]},我们想逐级展开这个序列,然后将所有元素装入一个序列。

如果这个序列层级较少,我们可以用多层\codeinline{python}{for}循环来遍历这个序列。一旦这个序列超过3层,过多的循环会让你很头疼。同样,这种多层级的问题我们可以用\textbf{递归}来解决。构建一个函数,这个函数能处理第一层的元素,由于第二层是\codeinline{python}{list},它是一个可迭代对象,我们只需要判断第二层是不是可迭代对象,同时忽略\codeinline{python}{str, bytes}对象\footnote{\codeinline{python}{str, bytes}也是可迭代对象,我们要避免其展开成单个字符。}。只要内层是可迭代的,我们就开始递归,对其应用该函数。

\pythonfile{code/python/unfold.py}

由于存在递归,所以函数会被调用很多次,每次调用所得的数据都需要保留,如何在多次的调用之间共享保留数据呢?我采用一个默认参数来实现\footnote{前几天没有回想起list有个extend方法,显然用extend方法来实现更加优雅},首次调用时不给默认参数新值,这会产生一个空的\codeinline{python}{list},当对内层对象调用时,将上一次产生的数据赋值给这个参数。输出结果:

\begin{pythoncode}
>>> items1 = ['Paula', ['Thomas', 'Lewis', ['siyu', 'ziyan', ['jianyuan']]]]
>>> items2 = [1, 2, [3, 4, [5, 6, [9, 8], 7], 8]]
>>> items3 = [[1, 2], 3, (4, [5, 6])]
>>> print(unfold(items1))
>>> print(unfold(items2))
>>> print(unfold(items3))
['Paula', 'Thomas', 'Lewis', 'siyu', 'ziyan', 'jianyuan']
[1, 2, 3, 4, 5, 6, 9, 8, 7, 8]
[1, 2, 3, 4, 5, 6]
\end{pythoncode}

但这样做有两个显而易见的坏处,一是当我们的嵌套序列有无限多层,递归会栈溢出;二是序列整个被读取到内存中了,当序列元素非常多,比如1亿,内存会被撑死。坏处一我们不去管他,大多数情况下是适用的,坏处二可以很容易的利用generator来解决\footnote{思路来源~\url{http://python3-cookbook.readthedocs.io/zh_CN/latest/c04/p14_flattening_nested_sequence.html}}。

\pythonfile{code/python/unfold_generator.py}

使用generator一来能防止内存爆炸,二来不需要在函数的多次调用见传递数据,代码更清晰明朗。需要注意,generator是惰性序列,边调用边计算,我们需要使用\codeinline{python}{for}迭代出每一个元素或者直接用\codeinline{python}{list()}获取全部元素。

\begin{pythoncode}
items1 = ['Paula', ['Thomas', 'Lewis', ['siyu', 'ziyan', ['jianyuan']]]]
items2 = [1, 2, [3, 4, [5, 6, [9, 8], 7], 8]]
items3 = [[1, 2], 3, (4, [5, 6])]
print(list(unfold(items1)))
print(list(unfold(items2)))
print(list(unfold(items3)))
\end{pythoncode}

\section{如何遍历当前文件夹及其子文件夹中的全部文件}

前面用MATLAB实现了一个,现在用Python来实现。第一种方法是利用递归来实现,思路同样是先找文件,然后找子文件夹,最后对子文件夹递归;第二种方法是利用os.walk模块,并将其做成generator,这样在应对大量的文件时会有优势。推荐第二种方法,一来os模块考虑了很多我们忽略了的细节\footnote{比如,如果递归版本的函数遍历的根目录是一个磁盘,这个磁盘上的特殊的文件夹“System Volume Information”又是禁止被访问的,这时就会抛出一个PermissionError。笨一点的解决办法是从子目录的list中删除这个目录,好一点的办法就是用os模块了。},二来generator是一个优雅的设计,用Python就应该好好学用generator。

\pythonfile{code/python/get_all_file_name.py}


