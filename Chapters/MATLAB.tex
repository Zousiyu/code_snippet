\chapter{MATLAB}

\section{如何遍历当前文件夹及其子文件夹中的全部文件}

假设现在我们有这样一个文件夹A,它含有一些文件和子文件夹B、C、D......我们需要将这个父文件夹(A)及其子文件夹(B、C、D......)中的所有文件名和其路径取出来。

如果你用的是MATLAB 2016b及其后面的版本,那真的太棒了!\Matlabinline|dir()|函数已经支持遍历搜索了。尝试敲入,

\begin{Matlabcode}
dir_data = dir('**/*');
dir_data([dir_data.isdir]) = [];  % 去除所有文件夹
\end{Matlabcode}

这将会返回一个包含文件信息的struct,现在你可以任意操作这些struct了,随意拼接路径。解放大脑,哦也!

方便归方便,但是,一来肯定有大多数人使用的是MATLAB 2016b之前的版本,二来,解放大脑意味着我失去了一次独立思考的机会。

\subsection*{思考}

对于实现方法\footnote{思路来源:\href{https://stackoverflow.com/questions/2652630/how-to-get-all-files-under-a-specific-directory-in-matlab}{How to get all files under a specific directory in MATLAB?}},多层次的遍历,我们第一时间想到的肯定是递归。其次是数据的存储,\Matlabinline|dir()|函数返回的是一个struct,我们要充分利用这个数据结构。所以现在思路是,写一个递归函数,这个函数返回包含所以文件的struct。

这个函数应对先处理父文件夹,获取文件和文件夹,然后获取子文件夹,我们对获取的子文件夹再次调用该函数,最后函数返回存储有所有文件信息的struct。现在,你可以对这个结构体做你想做的事情。

\subsection*{解}

\Matlabfile[firstline=2]{code/get_all_file_name.m}

\subsection*{关于效率}
\Matlabinline|dir()|函数遍历整个F盘文件大约需要1.555823s。我们实现的递归函数遍历F盘文件大约需要3.703009s。慢是慢了点,但我们成功运用了递归解决问题,不是吗?

\section{title}