\chapter{MATLAB}

\section{如何遍历当前文件夹及其子文件夹中的全部文件}

假设现在我们有这样一个文件夹A,它含有一些文件和子文件夹B、C、D......,这些子文件夹有包含若干层子文件夹。我们需要将这个父文件夹(A)及其子文件夹(B、C、D......)和孙文件夹中的所有文件名和其路径取出来。

如果你用的是MATLAB 2016b及其后面的版本,那真的太棒了!\Matlabinline|dir()|函数已经支持遍历搜索了。尝试敲入:

\begin{Matlabcode}
dir_data = dir('**/*');
dir_data([dir_data.isdir]) = [];  % 去除所有文件夹
\end{Matlabcode}

这将会返回一个包含文件信息的struct,现在你可以任意操作这些struct了,随意拼接路径。解放大脑,哦也!

方便归方便,但是,一来肯定有大多数人使用的是MATLAB 2016b之前的版本,二来,解放大脑意味着我失去了一次独立思考的机会。

\subsection*{思考}

对于实现方法\footnote{思路来源:\href{https://stackoverflow.com/questions/2652630/how-to-get-all-files-under-a-specific-directory-in-matlab}{How to get all files under a specific directory in MATLAB?}},多层次的遍历,我们第一时间想到的肯定是递归。然后就是数据的存储了,\Matlabinline|dir()|函数返回的是一个struct,我们要充分利用这个数据结构。所以现在思路是,写一个递归函数,这个函数返回包含所有文件的struct。

这个函数应对先处理父文件夹,获取文件和子文件夹,在获取子文件夹的过程中,我们需要去除`.'和`..'这两个特殊的文件夹。我们对获取的子文件夹再次调用该函数。如此,利用递归获取子子孙孙无穷尽文件夹的信息\footnote{其实这并不可能,因为递归是有栈高度限制的,调用函数压入栈,返回函数弹出栈,如果文件夹层次太深,一直压栈就会到达栈溢出警告的极限,例如Python的栈往往是100层,我想MATLAB的栈也大致如此,不会太高},最后函数返回存储有所有文件信息的struct。现在,你可以对这个结构体做你想做的事情。思路如\algoref{get_all_file_name}所示。

\begin{algorithm}[H]
\KwIn{path}
\KwOut{struct of file information}
\Function{get\_all\_file\_name(path)}{
    get file and sub\_dir information of current dir\;
    storing file information\;
    remove specific folder\;
    \BlankLine
    \BlankLine
    \For{first sub\_dir \KwTo last sub\_dir)}{
        get next sub\_dir\;
        \textbf{recursion} $ \Longrightarrow $ get\_all\_file\_name(path)\;}
%    \KwRet{struct of file information}\;
}
\caption{遍历获取当前文件夹及其所有子文件夹中的文件名}
\label{get_all_file_name}
\end{algorithm}

\subsection*{解}

代码仅与MATLAB 2016b以上的版本兼容,2016a及之前的版本dir struct信息并不完善,不是一个值得信赖的数据结构。

\Matlabfile[firstline=1]{code/get_all_file_name.m}

\subsection*{总结}
\Matlabinline|dir()|函数遍历整个F盘共2万余文件文件大约需要1.555823s。我们实现的递归函数遍历F盘文件大约需要3.703009s。慢是慢了点,但我们成功运用了递归解决问题,不是吗?

注:本代码在MATLAB 2017a下测试成功,运用了一些dir struct的特性,老版本报错请从struct部分Debug。

\section{title中英文标题}